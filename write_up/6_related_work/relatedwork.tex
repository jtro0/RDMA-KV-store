% this file is called up by thesis.tex
% content in this file will be fed into the main document

\chapter{Related Work}\label{ch:related-work} % top level followed by section, subsection


% ----------------------- paths to graphics ------------------------

% change according to folder and file names
\ifpdf
    \graphicspath{{7/figures/PNG/}{7/figures/PDF/}{7/figures/}}
\else
    \graphicspath{{7/figures/EPS/}{7/figures/}}
\fi


% ----------------------- contents from here ------------------------
%
\section{HERD}
There are several proposed RDMA KV store designs.
One of which, HERD \cite{kalia2014using}, uses a combination of transport types.
HERD uses RDMA \textit{WRITE} over UC for requests and \textit{SEND} over UD for responses.
Kalia et. al. have shown that incoming \textit{WRITE}s offers lower latency and higher throughput compared to \textit{READ}.
Outgoing \textit{WRITE}s have also been shown to not scale well, making this unadvisable for sending responses.

HERD out performs the RDMA KV store presented in this thesis significantly.
This could due to multiple factors:
\begin{itemize}
    \item HERD has implemented some optimizations towards prefetching before posting a \textit{SEND} WR.
    With this they have observed roughly 30\% improvement, in the best case comparison (2 random memory accesses at 4 CPU cores.)
    \item Making use of one-sided \textit{WRITE} for requests, bypassing the CPU.
    \item Using inlined data for key.
    Inlining small payloads decreases latency as there is no need for a DMA operation.
\end{itemize}

However, Kalia et. al. stated that performance should be comparable to \textit{SEND}/\textit{SEND}, given that inlining is possible.
It was found that for a large number of clients and/or requests, the round-robin polling used, is inefficient.
With 1000s of clients, a \textit{SEND}/\textit{SEND} design would scale better than the \textit{WRITE}/\textit{SEND} used in HERD.



% ---------------------------------------------------------------------------
% ----------------------- end of thesis sub-document ------------------------
% ---------------------------------------------------------------------------