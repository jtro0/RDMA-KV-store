
% this file is called up by thesis.tex
% content in this file will be fed into the main document

%: ----------------------- introduction file header -----------------------
\chapter{Introduction}

% the code below specifies where the figures are stored
%\ifpdf
%    \graphicspath{{1_introduction/figures/PNG/}{1_introduction/figures/PDF/}{1_introduction/figures/}}
%\else
%    \graphicspath{{1_introduction/figures/EPS/}{1_introduction/figures/}}
%\fi

% ----------------------------------------------------------------------
%: ----------------------- introduction content ----------------------- 
% ----------------------------------------------------------------------



%: ----------------------- HELP: latex document organisation
% the commands below help you to subdivide and organise your thesis
%    \chapter{}       = level 1, top level
%    \section{}       = level 2
%    \subsection{}    = level 3
%    \subsubsection{} = level 4
% note that everything after the percentage sign is hidden from output



\section{Context}
Seemingly ever growing tech giants, such as Facebook, Amazon and Google, require fast, reliable and scalable key-value storage (KV-store) to serve a variety of services: product recommendations, user preferences, and advertisements\cite{decandia2007dynamo,geambasu2010comet}.
Amazon's Dynamo is an example of such a highly available and used KV-store.
Dynamo is used by Amazon to provide its core services, such as session storage and shopping cart\cite{decandia2007dynamo}.
At peak, these services handle tens of millions customers a second.
This requires Dynamo to process requests in the span of a few milliseconds, to provide a consistent and fast user experience.

Remote direct memory access (RDMA) networks have increasingly become more popular in data centers.
Over time, RDMA has achieved higher throughput, lower latency, and lower cost of RDMA capapble network interface cards (RNIC)\cite{kalia2016design, chen2019scalable}.
RDMA and RNIC provides a lower latency networking interface compared to classical network cards, which is advantageous for latency sensitive applications, such as a KV-store, deployed in data centers.


\section{Problem statement}
Making efficient use of RDMA networks is a difficult task, and requires in-depth knowledge on the hardware constraints present in the RNICs\cite{kalia2016design, chen2019scalable}.
Additionally, in a higher level sense, there are design choices to be made, in regard to transportation types and so-called verbs.
RDMA networks can handle of various transportation types: reliable connection (RC), unreliable connection (UC) and unreliable datagrams (UD).
RDMA makes use of multiple verbs, which can be seen as network operations.
There are four main verbs: \textit{SEND}, \textit{RECV}, \textit{READ}, and \textit{WRITE}.
Each transportation type and verb come with advantages and disadvantages, which will be discussed further in this thesis.
However, impact of transportation types on scalability of an RDMA bases KV-store has not been explicitly examined.
Previous work has focused on researching verb choices, less on transportation type\cite{kalia2014using, kalia2016fasst, mitchell2013using}.

Scalability is an important factor for tech giants such as Amazon.
With the constant growth in customers and demand, for example for Amazon, services must be scalable to maintain performance with growing number of clients.

\section{Research Question}
This paper will explore to what extent RDMA transportation types affect the performance and multithreaded scalability of KV-store, under two designs: blocking and non-blocking.
For this, research will be conducted to answer the following questions:

\begin{itemize}
    \item[\textbf{RQ1}] What is the multithreaded scalability of the RDMA transport types: reliable connection (RC), unreliable connection (UC), and unreliable datagram (UD)?
    Scaling up by adding more resources to a system, in this case number of threads, is a more cost-effective strategy, compared to scaling horizontally by adding more compute nodes.
    For this reason, multithreaded scalability of these RDMA transport types will be investigated, with a blocking and non-blocking design.
    \item[\textbf{RQ2}] What are the advantages and disadvantages of RDMA transport types when implementing a RDMA KV-store?
    This question aims to aid with design process of future RDMA KV stores.
    By applying known ramifications of RDMA to that in a KV store use case.
    With supporting results, recommendations can be given, and possible design issues can be foreseen.
\end{itemize}

\section{Research Methods}
In this thesis the network performance of KV store, with varying network protocols, will be studied.
First an understanding of KV-stores is established.
Along with discussing known issues with traditional networking, RDMA will be introduced.
Experiments will be performed to measure overall throughput and latency, with increasing number of clients.
With this the possible scalability of each transport type, in both a blocking and non-blocking design, will be shown.

A prototype\cite{iosup2019atlarge,hamming1998art,peffers2007design} and experimental\cite{jain1990art,heiser2010,ousterhout2018always} approach is taken for this thesis:
\begin{itemize}
    \item[\textbf{M1}] A KV-store prototype will be implemented, with a flexible network interfacing, to include both traditional socket and RDMA interfacing.
    This, and all other relevant project files, are open source and can be found on Github\cite{github}.
    \item[\textbf{M2}] To investigate the performance and scalability between the various networking types, a workload realistic\cite{atikoglu2012workload} macro-level benchmark will be designed.
\end{itemize}

With the focus of this thesis being on the network implementation, a trivial KV store will be used.
This implementation does not offer strong performance and scalability, however these issues are minimized and kept consistent throughout experiments.

All measurements hereafter are recorded on the DAS-5 computing cluster.
This allows for a consistent working environment.
Further details on this is shown in table \ref{tab:das5} and discussed in section \ref{subsec:experimental-setup}

\section{Thesis Contributions}
This thesis presents scalability performance of RC, UC, and UD.
These are shown and compared to the well established TCP protocol.
Additionally, recommendation as to which transportation protocol is best suited for an RDMA KV-store application, and which design choices are available to improve scalability.
These findings can be used to further examine possible optimizations or RDMA verb choices, or aid in future RDMA KV store implementations.
The prototype and all project files can be found on Github\cite{github}.

\section{Plagiarism Declaration}
I herby declare that this thesis work, all data collected and findings are my own.

For this thesis, the basic structure of KV store assignment from course Operating Systems, at the VU, has been used, however, little of which has been kept.
Along with basic RDMA functions from Dr. Trivedi's RDMA example\cite{rdma_example}.
This has been used as a starting point for RC implementation.

\section{Thesis Structure}
Section \ref{ch:background} will provide the necessarily background knowledge of KV-store, linux sockets, and RDMA.
Next, section \ref{ch:design} will go over the design of the KV-store, networking interfacing, and benchmark.
In section \ref{ch:design} this design will be taken and described the implementation in a more technical prospective.
The benchmarking result will be presented and analyzed in section \ref{ch:evaluation}.
Section \ref{ch:related-work} compares findings with that of previously done work.
Future prospects will be given in section \ref{ch:future-improvements}.
Closing off the thesis, section \ref{ch:conclusion} will go over the conclusion and provide recommendations.