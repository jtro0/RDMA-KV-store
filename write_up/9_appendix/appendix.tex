% this file is called up by thesis.tex
% content in this file will be fed into the main document

%: ----------------------- name of chapter  -------------------------
\chapter{Appendix} % top level followed by section, subsection


%: ----------------------- paths to graphics ------------------------

% change according to folder and file names
\ifpdf
    \graphicspath{{X/figures/PNG/}{X/figures/PDF/}{X/figures/}}
\else
    \graphicspath{{X/figures/EPS/}{X/figures/}}
\fi

%: ----------------------- contents from here ------------------------
%\includepdf[pages=1-5]{CodesTable}
%\includepdf[pages=1-4]{InterviewQuestions}

\section{Reproducibility}
All project files can be found on Github\cite{github}.
A "README" file can be found there with some more details on how to build and configure DAS-5 and program environment.

\subsubsection{DAS-5 setup}
For these experiments, two DAS-5 nodes are reserved for 10800 seconds or 3 hours.
This is sufficient to run benchmarks for all transport types and TCP, up to 70 clients.
Make sure to use scratch folder on DAS, as this data can reach up to 43GB in data for the non-blocking design.
This should be linked inside benchmarking folder, which can be done by the following command: "ln -s /var/scratch/<das5\_username> benchmarking/data"
Data must be removed between blocking and non-blocking, as this would reach the limit of 50GB of scratch data.

\subsection{Server}
After compiling all binary executables as described in the "README", a server can be started.
A sever can be started by executing "./bin/kv\_server".
By default this starts a TCP server, however program arguments can be given, these are as follows:

\begin{itemize}
    \item --rdma -r <type> Use RDMA, requires type argument: rc, uc, ud
    \item --port -p <port> Change the port to a given port number. Default: 35304.
    \item --blocking Use blocking design, which waits for completion event.
    \item --help -h Print help message
    \item --verbose -v Print info messages.
    \item --debug -d Print debug messages
\end{itemize}

As example, to run a key-value server running UC and blocks until completion event: "./bin/kv\_server -r uc --blocking".

\subsection{Benchmark}
Similarly, benchmark is a binary executable, which can be started by "./bin/benchmark -a <server ip address>"
By default this runs with TCP, 10 million operations, and 1 client.
Again, the following arguments can be given:
\begin{itemize}
    \item -a <server ip address> IP address of the listening server.
    \item --clients -n <\# clients> Number of clients created.
    \item --ops -o <\# of operations Total number of operations executed among n-clients.
    \item --rdma -r <type> Use RDMA, requires type argument: rc, uc, ud
    \item --port -p <port> Change the port to a given port number. Default: 35304.
    \item --no-save Do not save data.
    \item --blocking Use blocking design, which waits for completion event.
    \item --help -h Print help message
    \item --verbose -v Print info messages.
    \item --debug -d Print debug messages
\end{itemize}

This will generate data and be stored in "benchmarking/data/blocking/" or "non\_blocking/".
This data can be used by the python scripts to draw graphs.

\subsection{Graphing}
There are 4 graphing scripts available in "graphing/": Average latency, latency box plot, latency CDF, and throughput.
Note, when running, current working directory must be in "graphing/"

\begin{itemize}
    \item Average latency can be run as follows: "python3 latency\_avg.py <max number of clients> blocking", with blocking is optional argument, and will use data in "benchmarking/data/blocking/". 
    This will generate a graph with the following filename: "benchmarking/graphs/Latency\_avg\_<max number of clients>\_blocking.pdf" (or without "\_blocking" if not given)
    \item Latency box plot can be run as follows: "python3 latency\_box.py <number of clients> blocking", where number of clients which client number shall be drawn.
    This will generate a graph with the following name: \\"benchmarking/graphs/Latency\_box\_no\_out\_<number of clients>\_blocking.pdf"
    \item Latency CDF plot can be run as follows: "python3 latency\_cdf.py <number of clients>", where number of clients which client number shall be drawn.
    This will generate a graph in "benchmarking/graphs/Latency\_cdf\_<number of clients>.pdf"
    \item Throughput can be run as follows: "python3 throughput.py <max number of clients> blocking".
    This will generate a graph in "benchmarking/graphs/Throughput\_<max number of clients>\_blocking.pdf"
\end{itemize}



% ---------------------------------------------------------------------------
%: ----------------------- end of thesis sub-document ------------------------
% ---------------------------------------------------------------------------

