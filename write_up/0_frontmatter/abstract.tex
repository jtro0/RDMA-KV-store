
% Thesis Abstract -----------------------------------------------------


%\begin{abstractslong}    %uncommenting this line, gives a different abstract heading
\begin{abstracts}        %this creates the heading for the abstract page

    Seemingly ever growing tech giants, such as Facebook, Amazon and Google, require fast, reliable and scalable key-value storage (KV-store) to serve product recommendations, user preferences, and advertisements.
    Past advancements of KV-stores have focussed on improving these underlying data structures.
    Remote direct memory access (RDMA) networks have increasingly become more popular in commercial and academic data centers.
    This relatively new technology offer lower latency and higher throughput performance compared to tradition sockets and network interface cards (NIC).
    Scalability performance of RDMA transportation protocols is less known, previous work focused on RDMA verb choice.

    This paper focuses on this gap, evaluating scalability performance of RDMA transportation types.
    These findings will be used to give recommendation for RDMA KV-store implementations.
    Macro-level benchmarks have been conducted, with realistic KV-store workloads, along with well established TCP performance.
    It has been found that RDMA offers a significant improvements compared to TCP.
    UD has shown to perform best, with 94.8\% throughput improvement over TCP, and 58.7\% improvement over RC.
    All the while offering consistent and gradual increase in latency up to 30 clients, reaching a maximum average latency of 0.06ms.
    Additionally, scalability issues of RC have been shown, and is not recommended in a KV-store application.

\end{abstracts}
%\end{abstractlongs}


% ---------------------------------------------------------------------- 
