
% Thesis Abstract -----------------------------------------------------


%\begin{abstractslong}    %uncommenting this line, gives a different abstract heading
\begin{abstracts}        %this creates the heading for the abstract page

    Seemingly ever growing tech giants, such as Facebook, Amazon and Google, require fast, reliable and scalable key-value storage (KV-store) to serve product recommendations, user preferences, and advertisements.
    A fast a reliable service is required to perform well under heavy workloads, with performance changes being unnoticed for the end user, such as a shopping customers at Amazon.
    The increase in number of customers and demand for such services, require data centers to scale resources, to maintain performance.
    Previously, KV-stores have maintained performance through improvements to the underlying data structure, typically a hash table.
    Remote Direct Memory Access (RDMA) networks, potentially gives rise to other KV store advancements.
    RDMA has become increasingly popular for data centers.
    The main advantage of RDMA, is that lower latency can be achieved.
    This is crucial in latency sensitive applications, such as KV-store.

    RDMA makes use of three so-called transport types: reliable connection (RC), unreliable connection (UC), and unreliable datagram (UD).
    Each transport type has its advantages and disadvantages.
    Previous work has not explicitly focused on transport type choice in an RDMA KV-store application.
    This paper focuses on this gap, evaluating multithreaded scalability performance of RDMA transportation types.
    These findings will be used to give recommendation for RDMA KV-store implementations.

    This paper made use of multithreaded KV server, with a worker thread for each additional client.
    Clients threads are created by a benchmark application, and perform a set number of random KV-store operations, in total 10 million operations.
    These operations are generated randomly, with a 95\% probability of a GET operation, and 5\% probability of SET operation.
    This has been used as a realistic workload, to perform macro-level benchmarks.
    Additionally, two designs have been evaluated, with and without blocking until completion event.
    This has been done, as large number of threads results in high CPU usage.

    After conducting experiements on the DAS-5 computing cluster,the results have shown that UC performs best out the three transport types.
    Without blocking, UC reaches a maximum throughput of 370 thousand ops/sec, at 32 clients.
    After 32 clients, the maximum physical threads have been reached, and performances drops below TCP, showing poor scalability.
    With blocking, UC stabilizes around 273 thousand ops/sec above 32 clients, showing no loss in performance with high number of threads.
    This loss in maximum performance has been shown to be caused by context switching.
    UC also achieves lower latency compared to TCP, in both blocking and non-blocking.
    However, the outliers in latency have been to shown to be significantly higher without blocking.
    Therefore, without any optimizations, it is recommended to use UC with blocking when exceeding maximum number of physical threads.
    UD performs close to UC without blocking, and does allow for more optimizations, thus UD is recommended with optimizations.

    All relevant project files can be found on Github\cite{github}.

\end{abstracts}
%\end{abstractlongs}


% ---------------------------------------------------------------------- 
